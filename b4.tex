\section*{B.4 Automation in der KFZ-Elektronik}

\subsection{Beschreiben Sie die Prozessschritte zur Herstellung eines Steuergerätes auf Leiterplatte. (S.75,79,80)}
1. SMD line top (Bestückung der Oberseite der LP) \\
2. SMD line bottom (Bestückung der Unterseite) \\
3. manual insertion (Bestückoperationen für nicht-SMD-Komponenten [Relais, Stecker, ..]) \\
4. assembly (Montage der LP mit Steckern und Gehäuse)

\subsection{Wie kann die Kennzeichnung eines Steuergerätes erfolgen? Wie kann sie elektronisch erfasst werden? Wozu kann die Kennzeichnung verwendet werden? (S.75, 81)}
Kennzeichnung per \emph{Laser Marking} (Nummer wird in LP ``eingebrannt'') oder Draufkleben einer
Etikette; entweder OCR (Text/Zahlen), Barcode oder Datenmatrix

Verwendung: Gerätenummer, Seriennummer, Produktionsdatum $\rightarrow$ traceability (zB: wenn eine
Serie eines IC's Probleme macht, der auf dem Steuergerät platziert wurde, können die Steuergeräte
``lokalisiert'' werden [jene Seriennummer sind bekannt, auf denen der fehlerhafte IC ist] und 
gegebenenfalls ausgetauscht werden)

\subsection{Beschreiben Sie die Prozessschritte an der Oberseite eines Steuergerätes auf Leiterplatte. (S.75,83-89)}
1. Lötpaste auftragen \\
2. Sichtkontrolle: Automatische optische Inspektion der Lötpaste (AOI) \\
3. Platzierung der SMD-Bauteile \\
4. Schmelzen der Lötpaste (Reflow Soldering) \\
5. Sichtkontrolle: AOI vom PCB (mit Kameras überprüfen ob: Bauteile überall draufsitzen und richtig 
sitzen, Lötqualität

\subsection{Beschreiben Sie die grundlegenden Lötverfahren zur Anbringung und Kontaktierung gehäuster Bauteile auf Leiterplatte und nennen Sie Anwendungsbeispiele. (S.76,97-98)}
\begin{description}
%\item[Reflow Löten (reflow soldering):]
\item[Schwall-Löten (wave soldering):] Flussmittel aufsprühen $\rightarrow$ vorheizen $\rightarrow$ 
  LP wird über eine Lötwelle gefahren - pumpt flüssiges Lot durch eine Öffnung in der Welle auf die 
  LP (Unterseite der LP wird somit verlötet)
\item[Minischwall-Löten (mini wave soldering):] ???
\end{description}

\subsection{Nennen Sie die Herstellverfahren beidseitig bestückter Steuergeräte auf Leiterplatte. (S.76,91-94)}
\begin{description}
\item[Reflow/Reflow:] die LP-Seiten werden seperat/hintereinander bestückt und das Reflow Soldering 
  durchgeführt (funktioniert bei ``leichten'' Bauteilen - das neuerliche Erwärmen der Unterseite im 
  Ofen, lässt die Bauteile nicht herunterfallen);
\item[Reflow/Wave:] eine LP-Seite wird mittels Reflow Soldering aufgetragen (Oberseite) 
  $\rightarrow$ Kleber für jedes Bauteil auf der anderen (Unter-)Seite auftragen (mit Roboterarm 
  oder Sieb) $\rightarrow$ bestücken $\rightarrow$ Kleber eventuell härten (durch den Ofen schieben) 
  und anschließend Schwall-Löten der Unterseite
\end{description}

\subsection{Beschreiben Sie alle Prüfschritte bei der Herstellung von Steuergeräten.}

\subsection{Wozu werden Verguss- und Klebeprozesse bei der Herstellung von Steuergerätesn eingesetzt? We kann Kleber bzw. Verguss aufgebracht werden?}

\subsection{Beschreiben Sie die Prozessschritte zur Herstellung eines Steuergerätes auf keramischen Schaltungsträgern.}

\subsection{Welche Vor- und Nachteile haben Steuergeräte auf keramischen Schaltungsträgern? Nennen Sie typische Anwendungsbeispiele in der Automobiltechnik.}

\subsection{Beschreiben Sie die mechanische und elektrische Anbindung von ungehäusten Halbleitern auf keramischen Schaltungsträgern.}
