\section*{B.4 Automation in der KFZ-Elektronik}

\subsection{Beschreiben Sie die Prozessschritte zur Herstellung eines Steuergerätes auf Leiterplatte.}

\subsection{Wie kann die Kennzeichnung eines Steuergerätes erfolgen? Wie kann sie elektronisch erfasst werden? Wozu ann die Kennzeichnung verwendet werden?}

\subsection{Beschreiben Sie die Prozessschritte an der Oberseite eines Steuergerätes auf Leiterplatte.}

\subsection{Beschreiben Sie die grundlegenden Lötverfahren zur Anbringung und Kontaktierung gehäuster Bauteile auf Leiterplatte und nennen Sie Anwendungsbeispiele.}

\subsection{Nennen Sie die Herstellverfahren beidseitig bestückter Steuergeräte auf Leiterplatte.}

\subsection{Beschreiben Sie alle Prüfschritte bei der Herstellung von Steuergeräten.}

\subsection{Wozu werden Verguss- und Klebeprozesse bei der Herstellung von Steuergerätesn eingesetzt? We kann Kleber bzw. Verguss aufgebracht werden?}

\subsection{Beschreiben Sie die Prozessschritte zur Herstellung eines Steuergerätes auf keramischen Schaltungsträgern.}

\subsection{Welche Vor- und Nachteile haben Steuergeräte auf keramischen Schaltungsträgern? Nennen Sie typische Anwendungsbeispiele in der Automobiltechnik.}

\subsection{Beschreiben Sie die mechanische und elektrische Anbindung von ungehäusten Halbleitern auf keramischen Schaltungsträgern.}
