\section*{B.4 Automation in der KFZ-Elektronik}

\subsection{Beschreiben Sie die Prozessschritte zur Herstellung eines Steuergerätes auf Leiterplatte. (S.75,79,80)}
1. SMD line top (Bestückung der Oberseite der LP) \\
2. SMD line bottom (Bestückung der Unterseite) \\
3. manual insertion (Bestückoperationen für nicht-SMD-Komponenten [Relais, Stecker, ..]) \\
4. assembly (Montage der LP mit Steckern und Gehäuse)

\subsection{Wie kann die Kennzeichnung eines Steuergerätes erfolgen? Wie kann sie elektronisch erfasst werden? Wozu kann die Kennzeichnung verwendet werden? (S.75, 81)}
Kennzeichnung per \emph{Laser Marking} (Nummer wird in LP ``eingebrannt'') oder Draufkleben einer
Etikette; entweder OCR (Text/Zahlen), Barcode oder Datenmatrix

Verwendung: Gerätenummer, Seriennummer, Produktionsdatum $\rightarrow$ traceability (zB: wenn eine
Serie eines IC's Probleme macht, der auf dem Steuergerät platziert wurde, können die Steuergeräte
``lokalisiert'' werden [jene Seriennummer sind bekannt, auf denen der fehlerhafte IC ist] und 
gegebenenfalls ausgetauscht werden)

\subsection{Beschreiben Sie die Prozessschritte an der Oberseite eines Steuergerätes auf Leiterplatte. (S.75,83-89)}
1. Lötpaste auftragen \\
2. Sichtkontrolle: Automatische optische Inspektion der Lötpaste (AOI) \\
3. Platzierung der SMD-Bauteile \\
4. Schmelzen der Lötpaste (Reflow Soldering) \\
5. Sichtkontrolle: AOI vom PCB (mit Kameras überprüfen ob: Bauteile überall draufsitzen und richtig 
sitzen, Lötqualität

\subsection{Beschreiben Sie die grundlegenden Lötverfahren zur Anbringung und Kontaktierung gehäuster Bauteile auf Leiterplatte und nennen Sie Anwendungsbeispiele. (S.76,97-98)}
\begin{description}
\item[Reflow Löten (reflow soldering):] siehe auch letzte Frage - Paste $\rightarrow$ Bauteile
  platzieren $\rightarrow$ schmelzen (durch den Ofen fahren)
\item[Schwall-Löten (wave soldering):] Flussmittel aufsprühen $\rightarrow$ vorheizen $\rightarrow$ 
  LP wird über eine Lötwelle gefahren - pumpt flüssiges Lot durch eine Öffnung in der Welle auf die 
  LP (Unterseite der LP wird somit verlötet) $\rightarrow$ gleichmäßige Benetzung der Oberfläche mit
  Lot
\item[Minischwall-Löten (mini wave soldering):] lokal begrenzt (anstatt einer Welle wird nur ein 
  ``Kamin'' verwendet) $\rightarrow$ Teile von einer LP verlöten, kein komplettes Bad (zB nur die 
  Stecker am Rand); meist keine optische Inspektion danach
\end{description}

\subsection{Nennen Sie die Herstellverfahren beidseitig bestückter Steuergeräte auf Leiterplatte. (S.76,91-94)}
\begin{description}
\item[Reflow/Reflow:] die LP-Seiten werden seperat/hintereinander bestückt und das Reflow Soldering 
  durchgeführt (funktioniert bei ``leichten'' Bauteilen - das neuerliche Erwärmen der Unterseite im 
  Ofen, lässt die Bauteile nicht herunterfallen);
\item[Reflow/Wave:] eine LP-Seite wird mittels Reflow Soldering aufgetragen (Oberseite) 
  $\rightarrow$ Kleber für jedes Bauteil auf der anderen (Unter-)Seite auftragen (mit Roboterarm 
  oder Sieb) $\rightarrow$ bestücken $\rightarrow$ Kleber eventuell härten (durch den Ofen schieben) 
  und anschließend Schwall-Löten der Unterseite
\end{description}

\subsection{Beschreiben Sie alle Prüfschritte bei der Herstellung von Steuergeräten. (S.76,99-101,108)}
\begin{description}
\item[Automatic Optical Inspection (AOI):] mit Kameras wird überprüft...

  - Bauteil vorhanden und Position \\
  - Lotqualität (zB: Kurzschluss zwischen 2 Lötstellen, keine/unzureichende Verlötung)

  meist konservativ eingestellt - bei möglichen Fehler(n) wird die LP ausgeworfen und muss von 
  Hand untersucht werden;
\item[In Circuit Test (ICT):] auf der LP werden Testpunkte vorgesehen oder Durchkontaktierungen 
  verwendet um mit einem Nadeladapter zu testen ob...

  - alle Komponenten platziert sind \\
  - Werte von Kondensatoren, Widerständen stimmen \\
  - Brücken, Verbindungen passen \\
  - traceability \\
  - programming / Funktion (grob) testen 
\item[Durchlauftunnelanlage (DTA):] finaler Stresstest und funktionaler Test - meist wird das Gerät 
  bei -40$^{\circ}$C und 85$^{\circ}$C und Raumtemperatur in Betrieb genommen um Frühausfälle (infant 
  mortality, siehe Badewannenkurve) auszusortieren
\end{description}

\subsection{Wozu werden Verguss- und Klebeprozesse bei der Herstellung von Steuergerätesn eingesetzt? Wie kann Kleber bzw. Verguss aufgebracht werden? (S.77,109-110,112)}
Verguss/Kleber: wasserdicht, Korrosionsschutz, ...

- complete potting (kompletter Verguss von Bauelementen auf LP mit dem Gehäuse) \\
- partly potting (nur teilweiser Verguss - LP und Gehäuse) \\
- Verguss einer Öffnung (``Loch flicken'') \\
- Verguss zur Vibrationsdämpfung (zB bei Beschleunigungssensoren im Airbag)

anstatt Kleben empfiehlt sich eine \emph{Orbital Friction Welding Connection} (= 2 Kunststoffteile
werden so aneinander gereibt, dass die Teile verschmelzen $\rightarrow$ + schnell, + stark, + nur
lokale Erwärmung);

\subsection{Beschreiben Sie die Prozessschritte zur Herstellung eines Steuergerätes auf keramischen Schaltungsträgern. (S.77-78,115-118,120-124)}
\begin{description}
\item[1. SMD line:] Silberleitkleber auftragen (Siebdruck) und mit SMD-Bauteile (zB: Kondensatoren) 
  bestücken

  auf die Unterseite werden meist die Widerstände mit einer Paste (hohe Toleranzen) aufgedruckt; um
  den Widerstandswert ``einzustellen'', werden die Widerstände gelasert (Widerstand erhöht sich) und 
  anschließend getrimmt (solange ``angeschnitten'' bis der Wert passt); \emph{= Laser Trimming of
    Printed Thick Film Circuit}
\item[2a. Die Bonding / Wire Bonding:] Kleber für IC's auftragen $\rightarrow$ Bestückung mittels
  Pick \& Place (vom Wafer auf Folie) $\rightarrow$ Verbindung der Bauelemente mit Schaltungsträger
  mittels Golddrahtbonden (Stecker, Leistungsstufen, ... werden mittels Dickdrahtbonden kontaktiert
  - meist Aluminiumdrähte)
\item[2b. Power Stage:] Bestückung der Leistungsendstufen, w.o. oder mit anderen Technologien 
  (für bessere thermische Anbindung, zB \emph{Direct Copper Bonding} - IC liegt auf besserem 
  Wärmeleiter KUPFER)
\item[3. assembly:] Montage wie bei Leiterplatten (oft Komplettverguss)
\end{description}
wie bei Leiterplatten werden zwischen den Schritten Prüfverfahren durchgeführt (AOI, ICT, ..)

\subsection{Welche Vor- und Nachteile haben Steuergeräte auf keramischen Schaltungsträgern? Nennen Sie typische Anwendungsbeispiele in der Automobiltechnik. (S.114)}
+ hohe Arbeitstemperaturen ($<$155$^{\circ}$C) \\
+ hält Vibrationen viel besser aus (LP würde brechen) \\
$\rightarrow$ + kann zB direkt an den Motor montiert werden \\
+ kleine Baugröße \\
+ gute thermische Leitfähigkeit (Wärme kann besser abgeführt werden) \\
+ kosteneffektiv, umweltfreundlich, hohe Innovationsrate, ...

- aufwendige Herstellung \\
- Bestücken von IC's ohne Package viel langsamer

\subsection{Beschreiben Sie die mechanische und elektrische Anbindung von ungehäusten Halbleitern auf keramischen Schaltungsträgern. (S.119)}
\begin{description}
\item[Wedge-Wedge bonding:] (Dickdrathbonden) Draht wird durch ein eigenes Werkzeug platziert und 
  mittels Bond-Werkzeug verschweißt;
\item[Ball-Wedge bonding:] (Dünndrahtbonden) das Bond-Werkzeug führt gleichzeitig den Draht mit; 
  Draht wird platziert $\rightarrow$ durch Reibung ``verschweißt'' und anschließend ``abgerissen'';
\end{description}
