\section*{B.2 Kostenmanagement in der Praxis}

\subsection{Beschreiben Sie wie und wann Produktkosten definiert werden und welche Ma\ss nahmen zur Beeinflussung der Produktkosten verwendet werden k\"onnen.}
In die Berechnung der Produktkosten m\"ussen nicht nur Material- und Herstellungskosten, sondern auch die Kosten der gesamten Prozesskette inkludiert werden.
Die Produktkosten sollten m\"oglichst fr\"uh festgelegt werden, klare, fr\"uh definierte Kostenziele erm\"oglichen auch mittelfristige Ma\ss nahmen.

Ma\ss nahmen: kA was er hier wissen will, m\"oglicherweise \ref{foo} und \ref{bar}.

\subsection{Was versteht man unter Target Costing? Wie wird Target Costing angewendet?}
\label{foo}
Skriptum, Seite 65:\\
Die Gewinne oder Verluste von Morgen werden Heute festgelegt, wer das Ziel nicht kennt, kann den Weg dorthin nicht finden.

Skriptum, Seite 61:
\begin{itemize}
	\item Der Wirkungsgrad kostengestaltender Ma\ss nahmen ist in der Konzeptphase am h\"ochsten.
	\item Klare, fr\"uhzeitig definierte Kostenziele erm\"oglichen auch mittelfristige Maßnahmen.
\end{itemize}

%Anstatt die Frage ``Was wird ein Produkt kosten?'' (Kosten + Gewinnzuschlag, Cost-Plus Calculation) zu beantworten wird beim
%Target Costing anderst herum vorgegangen:
%\begin{itemize}
%	\item Mit der Reverse Calculation wird ausgehend vom erlaubten Produktpreis auf die erlaubten Kosten f\"ur die Entwicklung zur\"uckgerechnet.
%	\item Die erlaubten Kosten werden dann mit dem ``Target Cost Splitting'' auf die einzelnen Entwicklungschritte und Komponenten aufgeteilt.
%\end{itemize}

\subsection{Beschreiben Sie Ma\ss nahmen zur Senkung der Kosten bzw. Steigerung der Effektivit\"at in Entwicklung und Produktion.}
\label{bar}
Ma\ss nahmen zur Kostensenkung sind ua.:
\begin{itemize}
	\item \"Ahnliche Herstellungsprozesse mit \"ahnlichem Material verwenden.
	\item Einige verl\"assliche Zulieferer statt mehreren Unbekannten.
	\item Generische L\"osungen um Expertenwissen wiederverwenden zu k\"onnen.

	\item Benchmarking:
		sich mit dem Rest der Branche vergleichen.
	\item Best practice sharing:
		Bew\"ahrte Methoden innerhalb einer gro\ss en Firma bekannt machen.
	\item Simultaneous engineering

	\item Mitarbeiter schulen, Teambuilding
\end{itemize}

Zur Steigerung der Effektivit\"at in Entwicklung und Produktion:
\begin{itemize}
	\item Brainstroming um m\"oglichst viele Ideen zu finden.
	\item Die besten Ideen aussortieren.
	\item Diese Ideen versuchen im Betrieb umzusetzen.
	\item Effekt der Ideen mittels Reporting feststellen.
\end{itemize}
