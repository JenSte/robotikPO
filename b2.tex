\section*{B.2 Kostenmanagement in der Praxis}

\subsection{Beschreiben Sie wie und wann Produktkosten definiert werden und welche Ma\ss nahmen zur Beeinflussung der Produktkosten verwendet werden k\"onnen.}

\subsection{Was versteht man unter Target Costing? Wie wird Target Costing angewendet?}
Anstatt die Frage ``Was wird ein Produkt kosten?'' (Kosten + Gewinnzuschlag, Cost-Plus Calculation) zu beantworten wird beim
Target Costing anderst herum vorgegangen:
\begin{itemize}
	\item Mit der Reverse Calculation wird ausgehend vom erlaubten Produktpreis auf die erlaubten Kosten f\"ur die Entwicklung zur\"uckgerechnet.
	\item Die erlaubten Kosten werden dann mit dem ``Target Cost Splitting'' auf die einzelnen Entwicklungschritte und Komponenten aufgeteilt.
\end{itemize}

\subsection{Beschreiben Sie Ma\ss nahmen zur Senkung der Kosten bzw. Steigerung der Effektivit\"at in Entwicklung und Produktion.}
Ma\ss nahmen zur Kostensenkung sind ua.:
\begin{itemize}
	\item \"Ahnliche Herstellungsprozesse mit \"ahnlichem Material verwenden.
	\item Einige verl\"assliche Zulieferer statt mehreren Unbekannten.
	\item Generische L\"osungen um Expertenwissen wiederverwenden zu k\"onnen.

	\item Benchmarking:
		sich mit dem Rest der Branche vergleichen.
	\item Best practice sharing:
		Bew\"ahrte Methoden innerhalb einer gro\ss en Firma bekannt machen.
	\item Simultaneous engineering

	\item Mitarbeiter schulen, Teambuilding
\end{itemize}

Zur Steigerung der Effektivit\"at in Entwicklung und Produktion:
\begin{itemize}
	\item Brainstroming um m\"oglichst viele Ideen zu finden.
	\item Die besten Ideen aussortieren.
	\item Diese Ideen versuchen im Betrieb umzusetzen.
	\item Effekt der Ideen mittels Reporting feststellen.
\end{itemize}
