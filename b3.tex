\section*{B.3 Simultaneous Engineering}

\subsection{Was versteht man unter Simultaneous Engineering? Nenne Sie die wesentlichen Vorteile und Probleme.}
``Simultaneous Engineering'' bezeichnet ein Entwicklungsmodel f\"ur technische Produkte das die Entwicklungszeit
verk\"urzen soll.
Statt sequenziellen werden dabei parallele Abl\"aufe verwendet; besonders nutzbringend ist die Parallelisierung
der Entwicklung und Produktionsmittelplanung und der Entwicklungschritte einzelner Produktkomponenten.

Vorteile:
\begin{itemize}
	\item Zeitersparnis
	\item Fehler k\"onnen fr\"uher erkannt und beseitigt werden, bevor sie in einer späteren Phase hohe Kosten verursachen.
\end{itemize}
Nachteile:
\begin{itemize}
	\item Sobald in einem Arbeitsablauf gen\"ugend Informationen bereit stehen, wird mit dem N\"achsten parallel
		begonnen. Dort wird dann allerdings nicht von Anfang an mit dem endg\"ultigen Informationsstand gearbeitet,
		was teilweise zu Mehrarbeit f\"uhrt. (Allerdings ist die Summe der Prozessschritte k\"urzer als beim sequenziellen Model.)
\end{itemize}
