\section*{B.1 Randbedingungen im Automobilbau}

\subsection{Beschreiben Sie die wichtigsten Randbedingungen im Automobilbau und stellen Sie die Konsequenzen f\"ur die Gestaltung des Produktes bzw. der Produktherstellung dar.(S.56-62)}
\begin{description}
  \item[Hohe Stückzahlen (\textgreater 100.000 Stk/a):]
    Die hohen Stückzahlen ermöglichen es die variablen Kosten durch
    einmalige Investtionen(Forschung/Entwicklung, Verbesserungen von
    Werkzeug und Produktionsanlagen) zu reduzieren. $\rightarrow$
    Weniger Materialverbrauch billigere Fertigung (Variable Kosten).
  \item[Steigende Funktionalität:]
    Zum größten teil nur erreichbar durch den Einsatz von Elektronik.
    Steigende Funktionalität frisst Ersparnisse der Rationalisierung
    auf. (Preisneveau immer gleich)
  \item[Sinkende Produktkosten der Komponenten und Höhere Zuverlässigkeit:]
    Reduzierung der Fertigungsschritte, der Anzahl der Teile und des Materialeinsatzes
  \item[Geringerer Bauraum der Komponenten :]
    Reduzierung des Materialeinsatzes
  \item[Niedrigeres Gewicht der Komponenten:]
    Reduzierung der Anzahl der Teile und des \emph{Materialeinsatzes}:

    \begin{itemize}
    \item Mechatronik im räumlichen Sinn
    \item Multifunktionalität
    \item Integration in Silizium
    \item Funktionssubstitution durch Elektronik
    \end{itemize}

    
  \item[Kürzere Time to Market:]
    Simultanious Engineering
\end{description}
