\section*{A.3 Toleranzausgleich, Sensorik, Bewegungsführung}

\subsection{Beschreiben Sie die zwei prinzipiellen Methoden zum Toleranzausgleich und 
  nennen Sie Beispiele. (S.42,45)}
\begin{description}
\item[aktiv:] Roboter nachjustieren/positionieren; Bewegungsführung; 

  Bsp. 1: Stift in eine
  Bohrung setzen $\rightarrow$ mit Sensor Bohrung \emph{suchen} (laufendes Erfassen während 
  Bewegung) oder \emph{messen} (Momentaufnahme); 

  Bsp. 2: Bauteil für LP-Bestückung aufnehmen
  $\rightarrow$ Momentaufnahme/Bild von Sauger und Bauteil machen (Position messen) und 
  anschließend zentrieren
\item[passiv:] Toleranz verursacht ``elastische Verformung''; nachgiebige Regelung; 

  Bsp.: Stift in eine Bohrung
  setzen $\rightarrow$ Werkstück so gestalten, dass die Toleranz eine Kraft hervorruft und 
  den Greifer \emph{passiv} bewegt (keine Ecken, sondern Schrägen - hier: angefaster Stift
  und Bohrung)
\end{description}

\subsection{Beschreiben Sie das Prinzip der Sensorführung von Industrierobotern. Unter
  welchen Bedingungen ist Sensorführung sinnvoll? Welche Probleme sind dabei zu beachten? 
  (S.42,50)}
Prinzip: Abweichung zwischen Effektor und Werkstück messen/suchen $\rightarrow$ Rückführung
an Steuerung $\rightarrow$ Auswertung und Regelung der Achsen/Antriebe

Sinnvoll wenn: Bewegung störanfällig oder komplex (Korrektur der Bahn mit Sensoren), 
große Werkstücktoleranzen, hohe Genauigkeitsanforderungen, biegeschlaffe Werkstücke, 
keine geordnete Bereitstellung möglich (vgl. Stapel von Scheiben für die Montage in ein
Fahrzeug), ... (S.50)

Probleme: - zusätzlicher Engineering-Aufwand (Regelung anstatt ``nur'' Steuerung); - keine
einheitlichen Sensorschnittstellen vorhanden; - hohe Sensorkosten; - geringe 
Systemverfügbarkeit; - mangelnde Fähigkeiten der Steuerungen zur Umsetzung von
Sensorsignalen in Bewegungsmodifikationen; - unzureichende dynamische Eigenschaften der 
Steuerungen
